\section{Introduction}
\label{sec:introduction}
To realize the vision of continuously expanding human abilities in robotics,
surely the visual system is of great importance.
Specifically the task of object recognition is essential to the humanoid robot.

The primate brain's ability to categorize and recognize objects could lead to
the idea that the task in its general sense is easy and therefore also easily
copied onto an artificial system ie. “Computer vision”. However, it is not.
Although the idea of implementing a computer vision system analogous to
the biological visual system in the primate brain has existed since the 1970's
there is still long way for such a system to be developed.

The theories proposed by neuroscientist David Marr had,
until recently been found difficult to implement due to lack of computational power \citep{kruger2013deep}.
Hence the main focus in general computer vision has been on designing individual
task oriented algorithms that perform analysis and recognition of such features as color, shape etc.
The specific task of object recognition through pixel-information might seem
easy as the human brain seems to do it instantly and effortlessly.
However, the seemingly unlimited amount of different objects in addition to variation of size
and posture in the 3D space is a great obstacle in the development of object recognition.
Imagine the simple concept of a cup. As humans we instantly recognize the use of it
even when we have never seen that specific type of cup,
regardless of size, shape, handle/no handle, color etc.

Here, we give an overview of the results presented in \citet{kruger2013deep} and \citet{fidler2009learning}
on the subjects of object recognition in computer vision systems using learned hierarichal networks.
